\section{Conclusiones}
Durante el desarrollo del Modelo y del Diagrama encontramos muchas dificultades inherentes a pasar de un problema del mundo real a un modelo computacional. Las leyes en general y el Estatuto Universitario en particular utilizan el castellano para establecer las reglas del dominio, y si bien intentan desambiguar las palabras y expresiones, no lo logran del todo. Además, es muy sencillo escribir restricciones en lenguaje natural que terminan teniendo un impacto grande en el modelo relacional, por ejemplo la restricción de los graduados en otras universidades nacionales, o la posibilidad de que haya empates en las elecciones.

De todos modos, descubrimos que el lenguaje "legal" (o que intenta serlo) se puede transformar a un modelo computacional de forma más sencilla que el lenguaje más cotidiano: gracias a sus intentos de dejar de lado ambigüedades y cubrir todos los casos posibles, se parece más a una especificación formal que lo que estamos acostumbrados a formalizar. Además, nuestro conocimiento personal acerca del dominio nos ayudó a cubrir los agujeros que el enunciado dejaba abiertos.

Por último, consideramos que este tipo de aprendizaje es importante para el mundo real, donde muchas veces es necesario modelar situaciones de este estilo o más complejas para poder armar programas que trabajen sobre las entidades.