\section{Introducción}
En este trabajo se presenta el modelado de una base de datos correspondiente a un problema del mundo real.

El problema encarado es el cogobierno de la Universidad de Buenos Aires. Se usó como insumo el enunciado del problema y el conocimiento específico que teníamos los integrantes del grupo sobre el Estatuto Universitario.

El objetivo principal de la base de datos es poder modelar las votaciones que se dan en los distintos órganos de cogobierno, incluyendo elecciones de Decanos y de Rector, guardando historia, resultados de las votaciones, y la información necesaria para saber quién votó cada propuesta en caso de las votaciones nominales, entre otras posibles consultas.