\documentclass[a4paper,10pt]{article}

\usepackage[margin=1in]{geometry} 	% Setea el margen manualmente, todos iguales.
\usepackage[spanish]{babel} 		% {Con estos dos anda
\usepackage[utf8]{inputenc} 		% todo lo que es tildes y ñ}
\usepackage{fancyhdr} 			%{Estos dos son para
\usepackage{ulem}
\pagestyle{fancyplain} 			% el header copado}
\usepackage{color}			% Con esto puedo hacer la matufia de poner en color blanco un texto para engañar al formato
\usepackage{graphicx}	% Para insertar gráficos
\usepackage{array}			% Para usar arrays
\usepackage{hyperref}		% Para que tenga links el índice
\usepackage{lscape}         % Para apaisar figuras
\usepackage{ textcomp }
\usepackage{ulem}

%\usepackage{datetime}	% Para agregar automáticamente fecha/hora de compilación y otras cosas

\lhead{Bases de Datos} 	% {Con esto se usa el header copado. También está \chead para
\rhead{TP1} 	% el centro y comandos para el pie de página, buscar fancyhdr}
\renewcommand{\footrulewidth}{0.4pt}
\lfoot{Facultad de Ciencias Exactas y Naturales}
\rfoot{Universidad de Buenos Aires}
%\rfoot{\textit{}}
\usepackage{amsfonts}	% para simbolos de reales, naturales, etc. se usa \mathbb{•} y la letra
\usepackage{amsmath}	% para \implies
%\usepackage{algorithm}
%\usepackage{algorithmic}
\usepackage{caratula}
\usepackage{pdfpages}

%%%%%%%%%%%%%%%%%%%%%%%%%%%%%%%%%%%%%
%      COMANDOS ÚTILES USADOS       %
%%%%%%%%%%%%%%%%%%%%%%%%%%%%%%%%%%%%%

% \section{title} 		Te hace un título ``importante'' en negrita, numerado. También está \subsection{title} y \subsubsection{title}.
% \begin{itemize}		Te hace viñetas.
%	\item esto es un item	Cambiar itemize por enumerate te hace una numeración.
% \end{itemize}

% \textbf{text} 		Te hace el texto en negrita (bold).
% \underline{text}		Te subraya el texto.

% \textsuperscript{text}	Te hace ``superindices'' con texto. En teoría subscript debería funcionar, pero se puede usar guion bajo entre llaves
% 				y signos peso para hacerlo como alternativa. Sino buscar.

% \begin{tabular}{cols} 	Es para hacer tablas. Se pone una c por cada columna deseada dentro de cols (si es que se desea centrada, l para justificar a 
%	a & b & c		izquierda, r a la derecha). Si se separa por espacios la tabla no tendrá líneas divisorias. Si se separa por | en lugar de 
% \end{tabular}			espacios, aparecerá una línea. Con || dos, y así. Luego para los elementos de las filas se escriben y se separan con ampersand (&).
%				Finalmente, para las líneas horizontales, se usa \hline para una linea en toda la tabla y \cline{i - j} te hace la linea desde
%				la celda i hasta la j, arrancando en 1.
%				Si en la columna se pone p(width) podés escribir un párrafo en la celda. Para hacer un enter con \\ no funciona porque te hace un
%				enter en la fila. Para eso se usa el comando \newline.
  
% \textcolor{color predefinido en palabras}{text}

%%%%%%%%%%%%%%%%%%%%%%%%%%%%%%%%%%%%%
%    FIN COMANDOS ÚTILES USADOS     %
%%%%%%%%%%%%%%%%%%%%%%%%%%%%%%%%%%%%%

\newcommand{\Gather}[1]{\begin{gather*}#1\end{gather*}}
%\newcommand{\Def}[1]{\textbf{Definición: }#1}
%\newcommand{\Prop}[1]{\textbf{Propiedad: }#1}
%\newcommand{\Teo}[1]{\textbf{Teorema: }#1}
\newcommand{\Obs}[1]{\textbf{Observación: }#1}
%\newcommand{\Amat}{A \in \mathbb{R}^{n\textnormal{x}n}}
\newcommand{\filtro}[1]{\textbf{\textit{#1}}}
\renewcommand{\labelenumii}{\theenumii}
\renewcommand{\theenumii}{\theenumi.\arabic{enumii}.}

\begin{document}

%%%%%%%%%%%%%%%%%%%%%%%%%%%
%			INICIO DE CARÁTULA			%
%%%%%%%%%%%%%%%%%%%%%%%%%%%

%\include{caratula}

\materia{Bases de Datos}
\submateria{Segundo Cuatrimestre de 2014}
\titulo{Trabajo Práctico }

\grupo{}
\integrante{Martinez Suñe, Agustín}{630/11}{aemartinez@dc.uba.ar}
\integrante{Iglesias, Axel}{79/10}{axeligl@gmail.com}
\integrante{Lascano, Nahuel}{476/11}{laski.nahuel@gmail.com}
\integrante{Artuso, Pablo}{282/11}{artusopablo@gmail.com}

\begin{titlepage}
\maketitle
\thispagestyle{empty}
\end{titlepage} 

%%%%%%%%%%%%%%%%%%%%%%%%%%%
%				FIN DE CARÁTULA			%
%%%%%%%%%%%%%%%%%%%%%%%%%%%

\tableofcontents
%\clearpage

%%%%%%%%%%%%%%%%%%%%%%%%%%%
%					DESARROLLO				%
%%%%%%%%%%%%%%%%%%%%%%%%%%%

\section{Introducción}
Con el avance de la tecnología y la capacidad de almacenamiento, las bases de datos tienden a almacenar cada vez más información. En un principio parece una ventaja, sin embargo, se puede inferir fácilmente que un almacenamiento masivo puede ocasionar problemas a la hora de realizar búsquedas. Mientras más grande es el espacio de búsqueda, mayor será el tiempo a invertir para lograr el objetivo.

El \textbf{optimizador de consultas}, componente clave del motor de base de datos, es el que combate el problema del tiempo antes mencionado. Su misión es, como bien explica su propio nombre, tratar de optimizar la forma de realizar la consulta con el fin de obtener la respuesta lo más rápido posible. Entre muchas otras cosas, el componente hace un fuerte uso de \textbf{estimadores} para poder llevar a cabo la optimización. Un \textbf{estimador} es un componente que brinda información aproximada de la cantidad de elementos que cumplen cierta condición (también llamada \textbf{selectividad}) comparativa contra una constante dada. Por lo tanto, el optimizador intentará reescribir la consulta de distintas maneras (manteniendo su valor semántico) y a través de distintas estimaciones, tomar una desición de cuál será la forma más eficiente de realizar la consulta. Si bien la precisión de la estimación debe ser lo más ajustada posible tiene también importancia la velocidad con la que se realiza, dado que este procedimiento se repetirá por cada consulta a la base. Es por esto que el estimador debe estar organizado de modo tal que pueda resolver rápidamente las consultas pedidas, quizás a cambio de tomarse más tiempo a la hora de inicializarlo o reiniciarlo. Del mismo modo, la elección del estimador afecta fuertemente la precisión de los resultados, por lo cual el conocimiento sobre la información específica de los elementos (distribución, dispersión, densidad, etc.) juega un papel determinante a la hora de la elección del estimador.

En el presente trabajo, detallamos tres implementaciones distintas de estimadores, mostrando sus diferentes comportamientos frente a distintas distribuciones, comparándolos y extrayendo conclusiones.
\section{Modelo Entidad Relación}
\subsection{Diseño}
A continuación se observa el modelo completo que diseñamos como solución al problema planteado.

\begin{figure}[h!]
  \centering
  \includegraphics[width=0.9\textwidth]{./images/der1}
  \caption{Primera parte del Modelo Entidad Relación}
  \label{fig:clases4}
\end{figure}

\begin{figure}[h!]
  \centering
  \includegraphics[width=1\textwidth]{./images/der2}
  \caption{Segunda parte del Modelo Entidad Relación}
  \label{fig:clases4}
\end{figure}
\newpage
\subsection{Consideraciones}
Estas son algunas consideraciones que tuvimos con respecto al enunciado del trabajo.

\begin{itemize}
\item  No es necesario tener la informacion de todos los integrantes de cada claustro ya que la votacion de consejeros directivos es secreta.
\item Al no modelar los integrantes de cada claustro no modelamos las condiciones que deben cumplir los candidatos a los diferentes cargos. Asumimos que las personas que estan cada una de esas tablas cumplen las condiciones.
\item No modelamos explicitamente cuando un candidato a rector es efectivamente elegido, sino que esta informacion se deduce de la cantidad de votos que recibio cada candidato. En este sentido tambien las elecciones de Rector que necesitaron mas de una votacion se deduce por las fechas en las que se junto la Asamblea a votar Rector y la cantidad de votos que saco cada uno.

\end{itemize}


\subsection{Restricciones}
Para que el modelo cumpla con lo estipulado en el estatuto, se deben tener en cuenta las siguientes restricciones:

\begin{itemize}
\item Una misma lista no puede estar en dos elecciones de distintas facultades.
\item Todas las resoluciones de Consejo Directivo son votadas por consejeros de la misma facultad.
\item M es 5 en la relación de votación de Consejero Directivo y Consejero Superior
\item Las resoluciones de Consejo Directivo son votadas por 4 Consejeros estudiantiles, 4 graduados y 8 profesores, correspondientes a la facultad y período de la resolución.
\item La composición del Consejo Directivo cumple lo especificado en el estatuto (por ejemplo para estudiantes,3 para la mayoría y 1 para la primer minoría si llega al 20\% de los votos).  

\end{itemize}

\subsection{Modelo Relacional}
Este es el MR correspondiente a MER presentado anteriormente. En él se pueden ver las distintas tablas necesarias para realizar la implementación.


\noindent \textbf{Decano}(\underline{DNI}, Periodo) \\
\textbf{ResolucionConsejoDirectivo}(\underline{nroResolucion}, Fecha, \dashuline{idDecanoQueVota}, \\\dashuline{idDecanoElegido})\\
\textbf{ResolucionConsejoSuperior}(\underline{nroResolucion}, Fecha)\\
\textbf{Asamblea}(\underline{idAsamblea}, fecha)\\
\textbf{ResolucionAsamblea}(\underline{nroResolucion}, \dashuline{idAsamblea, idCandidatoRector, idCandidatoVicerrector})\\
\textbf{CandidatoARector}(\underline{DNI}, periodo)\\
\textbf{Universidad}(\underline{idUniversidad}, nombre)\\
\textbf{ConsejeroDirectivo}(\underline{DNI}, Claustro, \dashuline{idUniversidad, idLista})\\
\textbf{ConsejeroSuperior}(\underline{DNI})\\
\textbf{Lista}(\underline{idLista}, Nro, Claustro)\\
\textbf{Facultad}(\underline{idFacultad}, Nombre)\\
\textbf{Eleccion}(\underline{idEleccion}, Periodo, CantDeVolantes, \dashuline{idFacultad})\\
\textbf{CandidatoAVicerrector}(\underline{DNI})\\
\textbf{VotaDecanoAsamblea}(\underline{idDecano, nroResolucion}, voto)\\
\textbf{VotaSuperiorAsamblea}(\underline{DNI}, voto)\\
\textbf{VotaDirectivoAsamblea}(\underline{DNI}, voto)\\
\textbf{VotaResolucionDirectivo}(\underline{DNI, nroResolucion}, voto)\\
\textbf{VotaResolucionSuperior}(\underline{DNI, nroResolucion}, voto)\\
\textbf{VotaDirectivoSuperior}(\underline{DNI\_Directivo, DNI\_Superior, periodo})\\
\textbf{Participa}(\underline{idLista, idEleccion})\\

\section{Conclusiones}
A lo largo del TP comprendimos que el problema de estimar la selectividad en una columna de una tabla no es trivial, sino que esconde todo un mundo de matemáticas, decisiones algoritmicas, de velocidad, de consumo de memoria de mayor o menor precisión. Implementamos diversos estimadores recomendados y diseñamos uno propio que logró mejor perfomance en la mayoría de las distribuciones provistas por la cátedra.

Analizamos la performance de los estimadores con diferentes distribuciones de datos, y concluimos que no hay un claro ganador para todas las distribuciones, sino que cada uno presenta ventajas y desventajas y que hace falta conocer el dominio en el que se van a utilizar para sacarles todo el jugo posible.

Descubrimos también que un estimador ``perfecto'' trae asociado siempre un gran costo, ya sea en velocidad o en memoria (o ambas), y que ni siquiera merece ser llamado estimador.

La mayor parte de nuestras predicciones teóricas se verificaron en la práctica, pero encontramos una situación curiosa según la cual aumentar en exceso un parámetro que pensábamos estaba directamente asociado con la ``precisión'' terminaba repercutiendo negativamente en la performance del estimador, dada su implementación algorítmica.

Como conclusión general de nuestras comparaciones, sostenemos que nuestro estimador se comporta mejor que los demás en el caso general, al menos para las distribuciones que analizamos, y preferentemente con un $S$ lo más grande que la memoria disponible permita. Solo priorizaríamos la elección de Steps si sabemos de antemano que las consultas presentarán más desigualdades que igualdades.


\end{document}