\section{Análisis empírico}

\subsection{Performance para diferentes distribuciones}
En esta sección calculamos la performance de los estimadores en datasets de diferentes distribuciones provistas por la cátedra y análizamos, en cada caso, mediante un test de hipótesis, si alguno de los estimadores es significativamente mejor o si, en términos estadísticos, no existe una diferencia importante de performance. 
El test utilizado es el \textit{Student’s T-Test Apareado} y para realizarlo necesitamos tener dos conjuntos de datos emparentados que representen la información que queremos comparar. En nuestro caso, para comparar el estimador $A$ con el estimador $B$ en cierto dataset, calculamos por un lado la performance de $A$ para estimar valores individuales del dataset y, por otro lado, la performance de $B$ para estimar esos mismos valores. Esto nos da dos listas de datos que están emparentadas por los valores que se estan estimando. Los valores a estimar cubren todo el rango de valores del dataset en intervalos de 10 unidades. Era necesario calcular la performance individual para diferentes valores del dataset y no simplemente tomar la performance como el promedio de todos esos valores ya que el valor promedio no es suficiente información para llevar adelante un test de hipótesis de estas características.

Presentamos a continuación los gráficos de los datasets utilizados:

\begin{figure}[h!]
  \centering
  \includegraphics[width=0.45\textwidth]{./images/c0}
  \includegraphics[width=0.45\textwidth]{./images/c1}
  \caption{Gráficos de los datasets de las columnas c0 y c1}
 \end{figure}
 
 
\begin{figure}[h!]
  \centering
  \includegraphics[width=0.45\textwidth]{./images/c2}
  \includegraphics[width=0.45\textwidth]{./images/c3}
  \caption{Gráficos de los datasets de las columnas c2 y c3}
 \end{figure}
 
\begin{figure}[h!]
  \centering
  \includegraphics[width=0.45\textwidth]{./images/c4}
  \includegraphics[width=0.45\textwidth]{./images/c5}
  \caption{Gráficos de los datasets de las columnas c4 y c5}
 \end{figure}
 
 
\begin{figure}[h!]
  \centering
  \includegraphics[width=0.45\textwidth]{./images/c6}
  \includegraphics[width=0.45\textwidth]{./images/c7}
  \caption{Gráficos de los datasets de las columnas c6 y c7}
 \end{figure}
 
 
\begin{figure}[h!]
  \centering
  \includegraphics[width=0.45\textwidth]{./images/c8}
  \includegraphics[width=0.45\textwidth]{./images/c9}
  \caption{Gráficos de los datasets de las columnas c8 y c9}
 \end{figure}
 
 