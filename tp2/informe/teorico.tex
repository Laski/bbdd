\section{Análisis teórico}
En esta sección realizamos un análsis teórico sobre la performance de los diferentes estimadores para tipos de distribuciones variados.

\subsection{Performance en distribuciones uniformes}
Para responder la pregunta de qué estimador debe exhibir mejor performance en un dataset con distribución uniforme, hace falta primero analizar con detenimiento este tipo de distribución. La distribución uniforme ``reparte'' los elementos en el rango disponible de forma equitativa: todos los enteros tienen la misma probabilidad de ocurrir.

Es decir, la forma de un Histograma Clásico de la distribución tendrá todas las barras del mismo ancho (por invariante de construcción del histograma) y de la misma altura, pues todos los rangos de idéntico ancho tienen aproximadamente la misma cantidad de elementos.

A su vez, la forma de un histograma de Distribution Steps tendrá todas las barras de la misma altura (por invariante de construcción del histograma) y del mismo ancho, pues la cantidad de elementos que resulta ser menor a un cierto porcentaje de los demás está relacionado directamente con ese porcentaje. Dicho en otras palabras, suponiendo un histograma de 10 barras, las barras tendrán un ancho aproximado al 10\% del rango de la distribución, pues eso mantiene la condición (de Distribution Steps) de que el límite derecho de la barra i deja a su izquierda 10i\% de los elementos (ver la sección de Estimadores para más información).

Entonces concluimos que, para un mismo parámetro (cantidad de barras) ambas distribuciones ``dibujarán'' prácticamente el mismo histograma. Esto podría inducir a pensar que la respuesta podría ser similar y que por lo tanto la performance será la misma, pero estaríamos olvidando que ``usan'' esos histogramas de maneras distintas.

\subsubsection{Consultas por igualdad}
Para las consultas por igualdad el Histograma Clásico computa la probabilidad dividiendo la altura de la barra en la que cae el número consultado por la cantidad de elementos totales. De este modo, si la cantidad de barras es 20, una distribución uniforme producirá una respuesta aproximada del 5\% (pues todas las barras tienen altura aproximadamente igual a 5\% del total). Sin embargo, si el rango es lo suficientemente grande, 5\% difícilmente sea una buena estimación (en particular para cualquier rango con más de 20 elementos cae por arriba de la probabilidad real). Concluimos entonces que la performance del Histograma Clásico será mejor cuanto más cercano sea el parámetro a la extensión del rango, pero de ser muy distintos estos valores (y en general lo son) puede provocar un error grande.

Por otro lado, el método de Distribution Steps (en su versión que minimiza el error en el caso promedio, que es la elegimos implementar) computa en el momento de su construcción el $\delta$ del que ya hablamos y responde basándose en ese valor. En particular en el caso más probable para una distribución uniforme (que el valor consultado coincida con a lo sumo un límite entre buckets, y no con uno extremo\footnote{Este caso es el más probable pues difícilmente en una distribución uniforme coincidan en valor varios límites de buckets: significaría que hay más elementos de ese valor de los que entran en dos buckets, pero para cantidades razonables (pequeñas) de buckets en relación al rango de la distribución es muy difícil que ocurra}) el estimador responde exactamente ese $\delta$. Es decir, antes que nada, que su performance difícilmente dependa de su parámetro (la cantidad de buckets) salvo que sea tan grande (en relación al rango de la distribución) como para que muchos elementos del mismo valor se extiendan a lo largo de varios buckets. Veamos entonces si ese $\delta$ es una buena estimación en el caso de una distribución uniforme.

$\delta$ es el mínimo entre $0.5/S$ y la densidad, donde $S$ es el parámetro (la cantidad de barras) y la densidad es la medida de la que ya hablamos en la sección de estimadores. Como ya dijimos, para valores razonables (no muy grandes) de $S$ es muy probable que la densidad sea menor a $0.5/S$, por lo cual en general caeremos en el caso $\delta = $ densidad. Como la densidad, informalmente hablando, es el porcentaje promedio de valores iguales, en el caso de una distribución uniforme será muy cercana a $1$ dividido la cantidad de elementos distintos, lo cual es exactamente la probabilidad de que un elemento de una distribución uniforme sea igual a un $x_0$ cualquiera. Por esto concluimos que independientemente del parámetro elegido, Distribution Steps estimará bien la selectividad por igualdad en una distribución uniforme.

Por último, nuestro estimador 

\subsubsection{Consultas por mayor}
Para las consultas por mayor el Histograma Clásico computa la selectividad haciendo el promedio entre dos probabilidades: la de que un elemento caiga en buckets mayores al $x_0$ (sumando las probabilidades de todos ellos) y la de que un elemento caiga en buckets mayores o en el mismo (sumando también las probabilidades de todos ellos), donde la ``probabilidad'' de un bucket es, como ya mencionamos, su altura sobre la cantidad de elementos.

Por su lado, el método de Distribution Steps presenta comportamientos matemáticamente más complejos, que dependen fuertemente de si el $x_0$ coincide con algún límite, con cuántos y con cuáles.





El caso de la estimación por mayor resulta en una distinto, pues en este caso el método de Distribution Steps sí utiliza la información del histograma.
