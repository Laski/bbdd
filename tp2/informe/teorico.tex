\section{Análisis teórico}
En esta sección realizamos un análsis teórico sobre la performance de los diferentes estimadores para tipos de distribuciones variados.

\subsection{Performance en distribuciones uniformes}
Para responder la pregunta de qué estimador debe exhibir mejor performance en un dataset con distribución uniforme, hace falta primero analizar con detenimiento este tipo de distribución. La distribución uniforme ``reparte'' los elementos en el rango disponible de forma equitativa: todos los enteros tienen la misma probabilidad de ocurrir.


En esta situación tanto el estimador basado en histograma clásico como el basado en Distribution Steps deberían exhibir la misma performance cuando son construidos con los mismos parámetros (cantidad de buckets y cantidad de steps respectivamente). Esto se debe, esencialmente, a que en el histograma de un distribución uniforme la cantidad de elementos que caen en cada bucket es, en términos probabilísticos, la misma. Esto quiere decir que los buckets no solo comparten la longitud sino que, en este caso, tienen la misma altura. Los steps del segundo estimador están distribuidos de tal manera que en cada intervalo se encuentra la misma cantidad de elementos, pero entonces por lo que vimos anteriormente, en el caso de la distribución uniforme coinciden con los buckets del histograma. Por lo tanto los estimadores no son significativamente diferentes.

